\documentclass[12pt]{article}
\usepackage[a4paper, total={6.5in,9.4in}]{geometry}

% special packages
\usepackage[utf8]{inputenc}
\usepackage{bm}
\usepackage[unicode]{hyperref}
\usepackage{amsmath,amssymb}
\usepackage{amsfonts}
\usepackage{enumitem}
\usepackage{natbib}
\usepackage{mathtools}
\usepackage{cancel}
\usepackage{empheq}
\usepackage{float}
\usepackage{nccmath}
\usepackage{subfig}


%boxes
\usepackage[most]{tcolorbox}
\usepackage[customcolors]{hf-tikz}

% define symbol names
\newcommand{\p}{\partial}
\newcommand{\vel}{\upsilon}
\usepackage{float}

% επιλογη γλωσσας.
\usepackage[greek,english]{babel}
\newcommand{\en}{\selectlanguage{english}}
\newcommand{\el}{\selectlanguage{greek}}

%μοναδιαίο διανυσμα
\newcommand{\uvec}[1]{\boldsymbol{\hat{{#1}}}}

%γραφικα
\usepackage{wrapfig}
\usepackage{graphicx}

% text styling
\setlength{\parindent}{0pt}

%boxes
\usepackage[most]{tcolorbox}
\usepackage[customcolors]{hf-tikz}


\begin{document}

\title{\en K-Means Clustering}
\el
\author{\en Tserpes Marios,
     AM:20029,
     \en mariostserp@econ.uoa.gr}

\en

\date{\today}

\maketitle

\tableofcontents

\newpage

\section{Exercise 1}

\begin{figure}[H] 
	\centering
	\subfloat[\en Exercise 1]{{\includegraphics[width=15.5cm] {1stexercise.jpg}}}
	\end{figure}
	
\paragraph{Step 1: Since we want to divide the data into 2 clusters we select 2 centers in a random way.}	

Therefore, I choose the following centroids.
\begin{equation*}
    & C_{1} = \binom {1}{1}\\
     C_{2} = \binom{2}{1}\\
\end{equation*}

\paragraph{Step 2: I'm calculating the distances between the points and the centroids.}

I create the following table:


\begin{center}
 \begin{tabular}{||c c c c c c||} 
 \hline
 A & B & C & D & Centroid1 & Centroid 2\\ [0.5ex] 
 \hline\hline
 1 & 2 & 4 & 5 & 1 & 2\\ 
 \hline
 1 & 1 & 3 & 4 & 1 & 1\\
 \hline
\end{tabular}
\end{center}

Using Euclidean Distance we will calculate the distance of the points from the CENTROID1.

\begin{align*}
    &dist(A, C_{1}) = \sqrt{(1-1)^2 +(1-1)^2} = \textbf{0}\\
    &dist(B, C_{1}) = \sqrt{(2-1)^2 + (1-1)^2}=\textbf{1}\\
    & dist(C, C_{1}) = \sqrt{(4-1)^2 + (3-1)^2} \approx \textbf{3.61}\\
    & dist(D, C_{1}) = \sqrt{(5-1)^2 + (4-1)^2} = \textbf{5}
\end{align*}

Using Euclidean Distance we will calculate the distance of the points from the CENTROID2.

\begin{align*}
    &dist(A, C_{2}) = \sqrt{(1-2)^2 +(1-1)^2} = \textbf{1}\\
    &dist(B, C_{2}) = \sqrt{(2-2)^2 + (1-1)^2}=\textbf{0}\\
    & dist(C, C_{2}) = \sqrt{(4-2)^2 + (3-1)^2} \approx \textbf{2.83}\\
    & dist(D, C_{2}) = \sqrt{(5-2)^2 + (4-1)^2} \approx \textbf{4.24}
\end{align*}

\paragraph{I will create a new distance table :}


In this step , we have to choose the shortest distance of points and centroids.
\begin{center}
 \begin{tabular}{||c c c c c c||} 
 \hline
 A & B & C & D & Centroid1 & Centroid 2\\ [0.5ex] 
 \hline\hline
 \textbf{0} & 1 & 3.61 & 5 & 1 & 2\\ 
 \hline
 1 & \textbf{0} & \textbf{2.83} & \textbf{4.24} & 1 & 1\\
 \hline
\end{tabular}
\end{center}

As we observe we have created new groups and in particular: In the first group we have only one point, while in the second group we have 3 points.

\paragraph{In this step we will try to reposition the points as to the initial centroids.}
\newline
With regard to Group 1 , after having a member , then it remains the same , i.e.
\begin{align*}
    C_{1} = \binom{1}{1}
\end{align*}

As far as Group 2 is concerned, we see that it has 3 members, i.e. points B, C, D, so I have to create the new centre, i.e.:

\begin{align*}
    & B = \binom{2}{1}, C = \binom{4}{3}, D = \binom{5}{4}\\
    & C_{2} = \binom{\frac{2+4+5}{3}}{\frac{1 + 3 + 4}{3}}= \binom{\frac{11}{3}}{\frac{8}{3}} \approx \binom{3.67}{2.67}\\
    & C_{2} = \binom{3.67}{2.67}\\
\end{align*}
\textbf{This is where the first iteration ends.}



\paragraph{Now, i will create the new distance table}
\begin{center}
 \begin{tabular}{||c c c c c c||} 
 \hline
 A & B & C & D & Centroid1 & New.Centroid 2\\ [0.5ex] 
 \hline\hline
 1 & 2 & 4 & 5 & 1 & 11/3 = 3.67\\ 
 \hline
 1 & 1 & 3 & 4 & 1 & 8/3 = 2.67\\
 \hline
\end{tabular}
\end{center}

\paragraph{Using Euclidean Distance i will compute the diastance of the points from the centroid 1.}

\begin{align*}
    &dist(A, C_{1}) = \sqrt{(1-1)^2 +(1-1)^2} = \textbf{0}\\
    &dist(B, C_{1}) = \sqrt{(2-1)^2 + (1-1)^2}=\textbf{1}\\
    & dist(C, C_{1}) = \sqrt{(4-1)^2 + (3-1)^2} \approx \textbf{3.61}\\
    & dist(D, C_{1}) = \sqrt{(5-1)^2 + (4-1)^2} = \textbf{5}
\end{align*}


\paragraph{Using Euclidean Distance i will compute the distance of the points from the centroid 2.}


\begin{align*}
    &dist(A, C_{2}) = \sqrt{(1-3.67)^2 +(1-2.67)^2} \approx \textbf{3.14}\\
    &dist(B, C_{2}) = \sqrt{(2-3.67)^2 + (1-2.67)^2} \approx \textbf{2.36}\\
    & dist(C, C_{2}) = \sqrt{(4-3.67)^2 + (3-2.67)^2} \approx \textbf{0.47}\\
    & dist(D, C_{2}) = \sqrt{(5-3.67)^2 + (4-3.67)^2} \approx \textbf{1.89}
\end{align*}

In this step , we have to choose the shortest distance of points and centroids.
\begin{center}
 \begin{tabular}{||c c c c c c||} 
 \hline
 A & B & C & D & Centroid1 & Centroid 2\\ [0.5ex] 
 \hline\hline
 \textbf{0} & \textbf{1} & 3.61 & 5 & 1 & 11/3\\ 
 \hline
 3.14  & 2.36 & \textbf{0.47} & \textbf{1.89} & 1 & 8/3\\
 \hline
\end{tabular}
\end{center}
\textbf{As we can see in group 1, two  members have been assigned (A, B) so the new center for group one is as follows, i.e:}

\begin{align*}
    & A = \binom{1}{1}, B = \binom{2}{1}\\
    & C_{1} = \binom{\frac{1 + 2}{2}}{\frac{1 + 1}{2}} = \binom{\frac{3}{2}}{\frac{2}{2}}\\
    & new.C_{1} = \binom{1.5}{1}
\end{align*}

\textbf{As we can see in group 2, two  members have been assigned (C, D) so the new center for group one is as follows, i.e:}

\begin{align*}
    & C = \binom{4}{3}, D = \binom{5}{4}\\
    & C_{2} = \binom{\frac{4 + 5}{2}}{\frac{3 + 4}{2}} = \binom{\frac{9}{2}}{\frac{7}{2}}\\
    & new.C_{2} = \binom{4.5}{3.5}
\end{align*}
\textbf{This is where the second and last iteration ends.This is where the repetition ends because the centres are now stopping changing.}
\newline
\paragraph{Therefore, the FINAL table can be illustrated as follow:}
\begin{center}
 \begin{tabular}{||c c c c c c||} 
 \hline
 A & B & C & D & First Centroid & Second Centroid\\ [0.5ex] 
 \hline\hline
 1 & 2 & 4 & 5 & 1.5 & 4.5\\ 
 \hline
  1 & 1 & 3 & 4& 1 & 3.5\\
 \hline
\end{tabular}
\end{center}

\textbf{VISUALLY:}
\begin{figure}[H] 
	\centering
	\subfloat[\en Exercise 1]{{\includegraphics[width=10.5cm] {1ex.jpg}}}
	\end{figure}

\end{document}
